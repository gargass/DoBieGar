% !TeX root = RJwrapper.tex
\title{Genes mutations}
\author{by Marlena Bielat, Małgorzata Dobkowska, Sebastian Gargas}

\maketitle


\subsection{Summary}\label{summary}

Application Genes mutations allows to browse the results of analysis of
mutations of genes from The Cancer Genome Atlas using RTCGA package. The
main goal is to find which biomarkers significantly impact on patient
survival in different types of cancers.

\subsection{Data}\label{data}

All of the used sets for the following cancers comes from RTCGA
repository:

\begin{itemize}
\itemsep1pt\parskip0pt\parsep0pt
\item
  GBMLGG - GBM + LGG (Glioblastoma multiforme and Lower Grade Glioma),
\item
  BRCA - Breast invasive carcinoma
\item
  KIPAN - Pan-kidney cohort
\item
  COADREAD - Colorectal adenocarcinoma
\item
  STES - Stomach and Esophageal carcinoma
\item
  GBM - Glioblastoma multiforme
\item
  OV - Ovarian serous cystadenocarcinoma
\item
  UCEC - Uterine Corpus Endometrial Carcinoma
\item
  KIRC - Kidney renal clear cell carcinoma
\item
  HNSC - Head-Neck Squamous Cell Carcinoma
\item
  LUAD - Lung adenocarcinoma
\item
  LGG - Lower Grade Glioma
\item
  LUSC - Lung squamous cell carcinoma
\item
  THCA - Thyroid carcinoma
\end{itemize}

The cancers from list corresponds to sets containing informations about at
least 500 patients. For each cancer we choose genes on which mutation
has occurred in at least 5\% cases and then we find genes which have
significant impact on patients survival. This significance was measured
by p-value of log-rank test comparing survival time in two groups of
patients divided by the presence of the mutation. The final list of
biomarkers is the union of significant genes for each type of cancer. In
our application we consider 535 different genes.

To analyze the impact of the gene mutation on survival time, we have
chosen two features:

\begin{itemize}
\itemsep1pt\parskip0pt\parsep0pt
\item
  presence of mutation - we divided patients in two groups: The first
  one refers to patients without the gene mutation. The second one
  refers to patients with at least one mutation on gene.
\item
  type of mutation - we used variable Variant Classification, in which
  mutations are classified into 10 different types:
  Missense\(\_\)Mutation, Silent, Frame\(\_\)Shift\(\_\)Del,
  Frame\(\_\)Shift\(\_\)Ins, In\(\_\)Frame\(\_\)Del,
  Nonsense\(\_\)Mutation, RNA, Splice\(\_\)Site, In\(\_\)Frame\(\_\)Ins,
  Nonstop\(\_\)Mutation.
\end{itemize}



\subsection{Summary of gene mutation}\label{summary-of-gene-mutation}

The table contains informations about the frequency and number of
patients with the mutation of the selected gene among patients suffering
from different types of cancers. It also includes information about the
significance of mutations to patients survival. It is measured by the
p-value of the log-rank test, which allow to compare the survival time
of two samples: in our case that samples are patients with the mutation
of the selected gene and patients without this mutation. The null
hypothesis is that the two groups have the same distribution of
survival time. If the p-value is less than 0.05 then we have grounds to
reject the null hypothesis at given significance level and the selected
gene can be considered as a prognostic biomarker.

\subsection{Survival curves: Presence of
mutation}\label{survival-curves-presence-of-mutation}

This tab displays Kaplan-Meier curves for the given gene and the given
cancers. The survival curves are estimated for the two groups of
patients: the first one refers to the patients with a mutation of the
given gene and the second one refers to the group of patients without
any mutation of this gene. The Kaplan-Meier method estimates the
probability of survival at least until specified on the X axis time.
Marks mean censored observations. The figures also show the p-value of
the log-rank test, which is helpful when we want to find out if two
survival curves can be considered as different. Genes mutations are rare
events, so estimates are often based on only a few cases - it is worth
to keep it in mind.

\subsection{Co-occurring genes}\label{co-occurring-genes}

The table presents informations about co-occurrence mutations. For the
selected gene and selected cancers a list of all considered biomarkers
is given with information about how often patients who have had a
mutation in a given gene have also mutation on listed genes. In
addition, we show number of patients suffering on specific types of
cancers with mutations on both genes.

\subsection{Survival curves: Variant
Classification}\label{survival-curves-variant-classification}

This tab also displays Kaplan-Meier curves for the given gene and the
given cancers for two classification variants - presence of missense
mutation and presence of nonsense mutation. Similar to tab `Survival
curves: Presence of mutation', the figures show the p-value of the
log-rank test, which examines the significance of a particular type of
mutation.

\subsection{Frequency of mutation
types}\label{frequency-of-mutation-types}

The table contains informations about the occurrence frequency of
different types of mutations for a selected gene and selected cancers.
As you can see in majority of cancers the most frequent types of
mutation are Missense Mutation and Nonsense Mutation. Groups with
another types of mutations offten are to small, so testing the
significance of a particular type of mutation is not justified. In
addition, we again show the level of occurrence of mutations in a given
gene.

\citep{R}

\bibliography{RJreferences}

\address{%
Marlena Bielat, \href{bielat.marlena@gmail.com}{\nolinkurl{bielat.marlena@gmail.com}}\\
Małgorzata Dobkowska, \href{gosia.dobkowska@tlen.pl}{\nolinkurl{gosia.dobkowska@tlen.pl}} \\
Sebastian Gargas, \href{gargass@student.mini.pw.edu.pl}{\nolinkurl{gargass@student.mini.pw.edu.pl}}\\
MiNI Warsaw University of Technology\\
\\
}



